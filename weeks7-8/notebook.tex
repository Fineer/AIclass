
% Default to the notebook output style

    


% Inherit from the specified cell style.




    
\documentclass[11pt]{article}

    
    
    \usepackage[T1]{fontenc}
    % Nicer default font (+ math font) than Computer Modern for most use cases
    \usepackage{mathpazo}

    % Basic figure setup, for now with no caption control since it's done
    % automatically by Pandoc (which extracts ![](path) syntax from Markdown).
    \usepackage{graphicx}
    % We will generate all images so they have a width \maxwidth. This means
    % that they will get their normal width if they fit onto the page, but
    % are scaled down if they would overflow the margins.
    \makeatletter
    \def\maxwidth{\ifdim\Gin@nat@width>\linewidth\linewidth
    \else\Gin@nat@width\fi}
    \makeatother
    \let\Oldincludegraphics\includegraphics
    % Set max figure width to be 80% of text width, for now hardcoded.
    \renewcommand{\includegraphics}[1]{\Oldincludegraphics[width=.8\maxwidth]{#1}}
    % Ensure that by default, figures have no caption (until we provide a
    % proper Figure object with a Caption API and a way to capture that
    % in the conversion process - todo).
    \usepackage{caption}
    \DeclareCaptionLabelFormat{nolabel}{}
    \captionsetup{labelformat=nolabel}

    \usepackage{adjustbox} % Used to constrain images to a maximum size 
    \usepackage{xcolor} % Allow colors to be defined
    \usepackage{enumerate} % Needed for markdown enumerations to work
    \usepackage{geometry} % Used to adjust the document margins
    \usepackage{amsmath} % Equations
    \usepackage{amssymb} % Equations
    \usepackage{textcomp} % defines textquotesingle
    % Hack from http://tex.stackexchange.com/a/47451/13684:
    \AtBeginDocument{%
        \def\PYZsq{\textquotesingle}% Upright quotes in Pygmentized code
    }
    \usepackage{upquote} % Upright quotes for verbatim code
    \usepackage{eurosym} % defines \euro
    \usepackage[mathletters]{ucs} % Extended unicode (utf-8) support
    \usepackage[utf8x]{inputenc} % Allow utf-8 characters in the tex document
    \usepackage{fancyvrb} % verbatim replacement that allows latex
    \usepackage{grffile} % extends the file name processing of package graphics 
                         % to support a larger range 
    % The hyperref package gives us a pdf with properly built
    % internal navigation ('pdf bookmarks' for the table of contents,
    % internal cross-reference links, web links for URLs, etc.)
    \usepackage{hyperref}
    \usepackage{longtable} % longtable support required by pandoc >1.10
    \usepackage{booktabs}  % table support for pandoc > 1.12.2
    \usepackage[inline]{enumitem} % IRkernel/repr support (it uses the enumerate* environment)
    \usepackage[normalem]{ulem} % ulem is needed to support strikethroughs (\sout)
                                % normalem makes italics be italics, not underlines
    

    
    
    % Colors for the hyperref package
    \definecolor{urlcolor}{rgb}{0,.145,.698}
    \definecolor{linkcolor}{rgb}{.71,0.21,0.01}
    \definecolor{citecolor}{rgb}{.12,.54,.11}

    % ANSI colors
    \definecolor{ansi-black}{HTML}{3E424D}
    \definecolor{ansi-black-intense}{HTML}{282C36}
    \definecolor{ansi-red}{HTML}{E75C58}
    \definecolor{ansi-red-intense}{HTML}{B22B31}
    \definecolor{ansi-green}{HTML}{00A250}
    \definecolor{ansi-green-intense}{HTML}{007427}
    \definecolor{ansi-yellow}{HTML}{DDB62B}
    \definecolor{ansi-yellow-intense}{HTML}{B27D12}
    \definecolor{ansi-blue}{HTML}{208FFB}
    \definecolor{ansi-blue-intense}{HTML}{0065CA}
    \definecolor{ansi-magenta}{HTML}{D160C4}
    \definecolor{ansi-magenta-intense}{HTML}{A03196}
    \definecolor{ansi-cyan}{HTML}{60C6C8}
    \definecolor{ansi-cyan-intense}{HTML}{258F8F}
    \definecolor{ansi-white}{HTML}{C5C1B4}
    \definecolor{ansi-white-intense}{HTML}{A1A6B2}

    % commands and environments needed by pandoc snippets
    % extracted from the output of `pandoc -s`
    \providecommand{\tightlist}{%
      \setlength{\itemsep}{0pt}\setlength{\parskip}{0pt}}
    \DefineVerbatimEnvironment{Highlighting}{Verbatim}{commandchars=\\\{\}}
    % Add ',fontsize=\small' for more characters per line
    \newenvironment{Shaded}{}{}
    \newcommand{\KeywordTok}[1]{\textcolor[rgb]{0.00,0.44,0.13}{\textbf{{#1}}}}
    \newcommand{\DataTypeTok}[1]{\textcolor[rgb]{0.56,0.13,0.00}{{#1}}}
    \newcommand{\DecValTok}[1]{\textcolor[rgb]{0.25,0.63,0.44}{{#1}}}
    \newcommand{\BaseNTok}[1]{\textcolor[rgb]{0.25,0.63,0.44}{{#1}}}
    \newcommand{\FloatTok}[1]{\textcolor[rgb]{0.25,0.63,0.44}{{#1}}}
    \newcommand{\CharTok}[1]{\textcolor[rgb]{0.25,0.44,0.63}{{#1}}}
    \newcommand{\StringTok}[1]{\textcolor[rgb]{0.25,0.44,0.63}{{#1}}}
    \newcommand{\CommentTok}[1]{\textcolor[rgb]{0.38,0.63,0.69}{\textit{{#1}}}}
    \newcommand{\OtherTok}[1]{\textcolor[rgb]{0.00,0.44,0.13}{{#1}}}
    \newcommand{\AlertTok}[1]{\textcolor[rgb]{1.00,0.00,0.00}{\textbf{{#1}}}}
    \newcommand{\FunctionTok}[1]{\textcolor[rgb]{0.02,0.16,0.49}{{#1}}}
    \newcommand{\RegionMarkerTok}[1]{{#1}}
    \newcommand{\ErrorTok}[1]{\textcolor[rgb]{1.00,0.00,0.00}{\textbf{{#1}}}}
    \newcommand{\NormalTok}[1]{{#1}}
    
    % Additional commands for more recent versions of Pandoc
    \newcommand{\ConstantTok}[1]{\textcolor[rgb]{0.53,0.00,0.00}{{#1}}}
    \newcommand{\SpecialCharTok}[1]{\textcolor[rgb]{0.25,0.44,0.63}{{#1}}}
    \newcommand{\VerbatimStringTok}[1]{\textcolor[rgb]{0.25,0.44,0.63}{{#1}}}
    \newcommand{\SpecialStringTok}[1]{\textcolor[rgb]{0.73,0.40,0.53}{{#1}}}
    \newcommand{\ImportTok}[1]{{#1}}
    \newcommand{\DocumentationTok}[1]{\textcolor[rgb]{0.73,0.13,0.13}{\textit{{#1}}}}
    \newcommand{\AnnotationTok}[1]{\textcolor[rgb]{0.38,0.63,0.69}{\textbf{\textit{{#1}}}}}
    \newcommand{\CommentVarTok}[1]{\textcolor[rgb]{0.38,0.63,0.69}{\textbf{\textit{{#1}}}}}
    \newcommand{\VariableTok}[1]{\textcolor[rgb]{0.10,0.09,0.49}{{#1}}}
    \newcommand{\ControlFlowTok}[1]{\textcolor[rgb]{0.00,0.44,0.13}{\textbf{{#1}}}}
    \newcommand{\OperatorTok}[1]{\textcolor[rgb]{0.40,0.40,0.40}{{#1}}}
    \newcommand{\BuiltInTok}[1]{{#1}}
    \newcommand{\ExtensionTok}[1]{{#1}}
    \newcommand{\PreprocessorTok}[1]{\textcolor[rgb]{0.74,0.48,0.00}{{#1}}}
    \newcommand{\AttributeTok}[1]{\textcolor[rgb]{0.49,0.56,0.16}{{#1}}}
    \newcommand{\InformationTok}[1]{\textcolor[rgb]{0.38,0.63,0.69}{\textbf{\textit{{#1}}}}}
    \newcommand{\WarningTok}[1]{\textcolor[rgb]{0.38,0.63,0.69}{\textbf{\textit{{#1}}}}}
    
    
    % Define a nice break command that doesn't care if a line doesn't already
    % exist.
    \def\br{\hspace*{\fill} \\* }
    % Math Jax compatability definitions
    \def\gt{>}
    \def\lt{<}
    % Document parameters
    \title{prac4}
    
    
    

    % Pygments definitions
    
\makeatletter
\def\PY@reset{\let\PY@it=\relax \let\PY@bf=\relax%
    \let\PY@ul=\relax \let\PY@tc=\relax%
    \let\PY@bc=\relax \let\PY@ff=\relax}
\def\PY@tok#1{\csname PY@tok@#1\endcsname}
\def\PY@toks#1+{\ifx\relax#1\empty\else%
    \PY@tok{#1}\expandafter\PY@toks\fi}
\def\PY@do#1{\PY@bc{\PY@tc{\PY@ul{%
    \PY@it{\PY@bf{\PY@ff{#1}}}}}}}
\def\PY#1#2{\PY@reset\PY@toks#1+\relax+\PY@do{#2}}

\expandafter\def\csname PY@tok@w\endcsname{\def\PY@tc##1{\textcolor[rgb]{0.73,0.73,0.73}{##1}}}
\expandafter\def\csname PY@tok@c\endcsname{\let\PY@it=\textit\def\PY@tc##1{\textcolor[rgb]{0.25,0.50,0.50}{##1}}}
\expandafter\def\csname PY@tok@cp\endcsname{\def\PY@tc##1{\textcolor[rgb]{0.74,0.48,0.00}{##1}}}
\expandafter\def\csname PY@tok@k\endcsname{\let\PY@bf=\textbf\def\PY@tc##1{\textcolor[rgb]{0.00,0.50,0.00}{##1}}}
\expandafter\def\csname PY@tok@kp\endcsname{\def\PY@tc##1{\textcolor[rgb]{0.00,0.50,0.00}{##1}}}
\expandafter\def\csname PY@tok@kt\endcsname{\def\PY@tc##1{\textcolor[rgb]{0.69,0.00,0.25}{##1}}}
\expandafter\def\csname PY@tok@o\endcsname{\def\PY@tc##1{\textcolor[rgb]{0.40,0.40,0.40}{##1}}}
\expandafter\def\csname PY@tok@ow\endcsname{\let\PY@bf=\textbf\def\PY@tc##1{\textcolor[rgb]{0.67,0.13,1.00}{##1}}}
\expandafter\def\csname PY@tok@nb\endcsname{\def\PY@tc##1{\textcolor[rgb]{0.00,0.50,0.00}{##1}}}
\expandafter\def\csname PY@tok@nf\endcsname{\def\PY@tc##1{\textcolor[rgb]{0.00,0.00,1.00}{##1}}}
\expandafter\def\csname PY@tok@nc\endcsname{\let\PY@bf=\textbf\def\PY@tc##1{\textcolor[rgb]{0.00,0.00,1.00}{##1}}}
\expandafter\def\csname PY@tok@nn\endcsname{\let\PY@bf=\textbf\def\PY@tc##1{\textcolor[rgb]{0.00,0.00,1.00}{##1}}}
\expandafter\def\csname PY@tok@ne\endcsname{\let\PY@bf=\textbf\def\PY@tc##1{\textcolor[rgb]{0.82,0.25,0.23}{##1}}}
\expandafter\def\csname PY@tok@nv\endcsname{\def\PY@tc##1{\textcolor[rgb]{0.10,0.09,0.49}{##1}}}
\expandafter\def\csname PY@tok@no\endcsname{\def\PY@tc##1{\textcolor[rgb]{0.53,0.00,0.00}{##1}}}
\expandafter\def\csname PY@tok@nl\endcsname{\def\PY@tc##1{\textcolor[rgb]{0.63,0.63,0.00}{##1}}}
\expandafter\def\csname PY@tok@ni\endcsname{\let\PY@bf=\textbf\def\PY@tc##1{\textcolor[rgb]{0.60,0.60,0.60}{##1}}}
\expandafter\def\csname PY@tok@na\endcsname{\def\PY@tc##1{\textcolor[rgb]{0.49,0.56,0.16}{##1}}}
\expandafter\def\csname PY@tok@nt\endcsname{\let\PY@bf=\textbf\def\PY@tc##1{\textcolor[rgb]{0.00,0.50,0.00}{##1}}}
\expandafter\def\csname PY@tok@nd\endcsname{\def\PY@tc##1{\textcolor[rgb]{0.67,0.13,1.00}{##1}}}
\expandafter\def\csname PY@tok@s\endcsname{\def\PY@tc##1{\textcolor[rgb]{0.73,0.13,0.13}{##1}}}
\expandafter\def\csname PY@tok@sd\endcsname{\let\PY@it=\textit\def\PY@tc##1{\textcolor[rgb]{0.73,0.13,0.13}{##1}}}
\expandafter\def\csname PY@tok@si\endcsname{\let\PY@bf=\textbf\def\PY@tc##1{\textcolor[rgb]{0.73,0.40,0.53}{##1}}}
\expandafter\def\csname PY@tok@se\endcsname{\let\PY@bf=\textbf\def\PY@tc##1{\textcolor[rgb]{0.73,0.40,0.13}{##1}}}
\expandafter\def\csname PY@tok@sr\endcsname{\def\PY@tc##1{\textcolor[rgb]{0.73,0.40,0.53}{##1}}}
\expandafter\def\csname PY@tok@ss\endcsname{\def\PY@tc##1{\textcolor[rgb]{0.10,0.09,0.49}{##1}}}
\expandafter\def\csname PY@tok@sx\endcsname{\def\PY@tc##1{\textcolor[rgb]{0.00,0.50,0.00}{##1}}}
\expandafter\def\csname PY@tok@m\endcsname{\def\PY@tc##1{\textcolor[rgb]{0.40,0.40,0.40}{##1}}}
\expandafter\def\csname PY@tok@gh\endcsname{\let\PY@bf=\textbf\def\PY@tc##1{\textcolor[rgb]{0.00,0.00,0.50}{##1}}}
\expandafter\def\csname PY@tok@gu\endcsname{\let\PY@bf=\textbf\def\PY@tc##1{\textcolor[rgb]{0.50,0.00,0.50}{##1}}}
\expandafter\def\csname PY@tok@gd\endcsname{\def\PY@tc##1{\textcolor[rgb]{0.63,0.00,0.00}{##1}}}
\expandafter\def\csname PY@tok@gi\endcsname{\def\PY@tc##1{\textcolor[rgb]{0.00,0.63,0.00}{##1}}}
\expandafter\def\csname PY@tok@gr\endcsname{\def\PY@tc##1{\textcolor[rgb]{1.00,0.00,0.00}{##1}}}
\expandafter\def\csname PY@tok@ge\endcsname{\let\PY@it=\textit}
\expandafter\def\csname PY@tok@gs\endcsname{\let\PY@bf=\textbf}
\expandafter\def\csname PY@tok@gp\endcsname{\let\PY@bf=\textbf\def\PY@tc##1{\textcolor[rgb]{0.00,0.00,0.50}{##1}}}
\expandafter\def\csname PY@tok@go\endcsname{\def\PY@tc##1{\textcolor[rgb]{0.53,0.53,0.53}{##1}}}
\expandafter\def\csname PY@tok@gt\endcsname{\def\PY@tc##1{\textcolor[rgb]{0.00,0.27,0.87}{##1}}}
\expandafter\def\csname PY@tok@err\endcsname{\def\PY@bc##1{\setlength{\fboxsep}{0pt}\fcolorbox[rgb]{1.00,0.00,0.00}{1,1,1}{\strut ##1}}}
\expandafter\def\csname PY@tok@kc\endcsname{\let\PY@bf=\textbf\def\PY@tc##1{\textcolor[rgb]{0.00,0.50,0.00}{##1}}}
\expandafter\def\csname PY@tok@kd\endcsname{\let\PY@bf=\textbf\def\PY@tc##1{\textcolor[rgb]{0.00,0.50,0.00}{##1}}}
\expandafter\def\csname PY@tok@kn\endcsname{\let\PY@bf=\textbf\def\PY@tc##1{\textcolor[rgb]{0.00,0.50,0.00}{##1}}}
\expandafter\def\csname PY@tok@kr\endcsname{\let\PY@bf=\textbf\def\PY@tc##1{\textcolor[rgb]{0.00,0.50,0.00}{##1}}}
\expandafter\def\csname PY@tok@bp\endcsname{\def\PY@tc##1{\textcolor[rgb]{0.00,0.50,0.00}{##1}}}
\expandafter\def\csname PY@tok@fm\endcsname{\def\PY@tc##1{\textcolor[rgb]{0.00,0.00,1.00}{##1}}}
\expandafter\def\csname PY@tok@vc\endcsname{\def\PY@tc##1{\textcolor[rgb]{0.10,0.09,0.49}{##1}}}
\expandafter\def\csname PY@tok@vg\endcsname{\def\PY@tc##1{\textcolor[rgb]{0.10,0.09,0.49}{##1}}}
\expandafter\def\csname PY@tok@vi\endcsname{\def\PY@tc##1{\textcolor[rgb]{0.10,0.09,0.49}{##1}}}
\expandafter\def\csname PY@tok@vm\endcsname{\def\PY@tc##1{\textcolor[rgb]{0.10,0.09,0.49}{##1}}}
\expandafter\def\csname PY@tok@sa\endcsname{\def\PY@tc##1{\textcolor[rgb]{0.73,0.13,0.13}{##1}}}
\expandafter\def\csname PY@tok@sb\endcsname{\def\PY@tc##1{\textcolor[rgb]{0.73,0.13,0.13}{##1}}}
\expandafter\def\csname PY@tok@sc\endcsname{\def\PY@tc##1{\textcolor[rgb]{0.73,0.13,0.13}{##1}}}
\expandafter\def\csname PY@tok@dl\endcsname{\def\PY@tc##1{\textcolor[rgb]{0.73,0.13,0.13}{##1}}}
\expandafter\def\csname PY@tok@s2\endcsname{\def\PY@tc##1{\textcolor[rgb]{0.73,0.13,0.13}{##1}}}
\expandafter\def\csname PY@tok@sh\endcsname{\def\PY@tc##1{\textcolor[rgb]{0.73,0.13,0.13}{##1}}}
\expandafter\def\csname PY@tok@s1\endcsname{\def\PY@tc##1{\textcolor[rgb]{0.73,0.13,0.13}{##1}}}
\expandafter\def\csname PY@tok@mb\endcsname{\def\PY@tc##1{\textcolor[rgb]{0.40,0.40,0.40}{##1}}}
\expandafter\def\csname PY@tok@mf\endcsname{\def\PY@tc##1{\textcolor[rgb]{0.40,0.40,0.40}{##1}}}
\expandafter\def\csname PY@tok@mh\endcsname{\def\PY@tc##1{\textcolor[rgb]{0.40,0.40,0.40}{##1}}}
\expandafter\def\csname PY@tok@mi\endcsname{\def\PY@tc##1{\textcolor[rgb]{0.40,0.40,0.40}{##1}}}
\expandafter\def\csname PY@tok@il\endcsname{\def\PY@tc##1{\textcolor[rgb]{0.40,0.40,0.40}{##1}}}
\expandafter\def\csname PY@tok@mo\endcsname{\def\PY@tc##1{\textcolor[rgb]{0.40,0.40,0.40}{##1}}}
\expandafter\def\csname PY@tok@ch\endcsname{\let\PY@it=\textit\def\PY@tc##1{\textcolor[rgb]{0.25,0.50,0.50}{##1}}}
\expandafter\def\csname PY@tok@cm\endcsname{\let\PY@it=\textit\def\PY@tc##1{\textcolor[rgb]{0.25,0.50,0.50}{##1}}}
\expandafter\def\csname PY@tok@cpf\endcsname{\let\PY@it=\textit\def\PY@tc##1{\textcolor[rgb]{0.25,0.50,0.50}{##1}}}
\expandafter\def\csname PY@tok@c1\endcsname{\let\PY@it=\textit\def\PY@tc##1{\textcolor[rgb]{0.25,0.50,0.50}{##1}}}
\expandafter\def\csname PY@tok@cs\endcsname{\let\PY@it=\textit\def\PY@tc##1{\textcolor[rgb]{0.25,0.50,0.50}{##1}}}

\def\PYZbs{\char`\\}
\def\PYZus{\char`\_}
\def\PYZob{\char`\{}
\def\PYZcb{\char`\}}
\def\PYZca{\char`\^}
\def\PYZam{\char`\&}
\def\PYZlt{\char`\<}
\def\PYZgt{\char`\>}
\def\PYZsh{\char`\#}
\def\PYZpc{\char`\%}
\def\PYZdl{\char`\$}
\def\PYZhy{\char`\-}
\def\PYZsq{\char`\'}
\def\PYZdq{\char`\"}
\def\PYZti{\char`\~}
% for compatibility with earlier versions
\def\PYZat{@}
\def\PYZlb{[}
\def\PYZrb{]}
\makeatother


    % Exact colors from NB
    \definecolor{incolor}{rgb}{0.0, 0.0, 0.5}
    \definecolor{outcolor}{rgb}{0.545, 0.0, 0.0}



    
    % Prevent overflowing lines due to hard-to-break entities
    \sloppy 
    % Setup hyperref package
    \hypersetup{
      breaklinks=true,  % so long urls are correctly broken across lines
      colorlinks=true,
      urlcolor=urlcolor,
      linkcolor=linkcolor,
      citecolor=citecolor,
      }
    % Slightly bigger margins than the latex defaults
    
    \geometry{verbose,tmargin=1in,bmargin=1in,lmargin=1in,rmargin=1in}
    
    

    \begin{document}
    
    
    \maketitle
    
    

    
    \section{Introduction to Artificial Intelligence for Non
Computing}\label{introduction-to-artificial-intelligence-for-non-computing}

    \subsection{Practical 4 (weeks 7 - 8)}\label{practical-4-weeks-7---8}

    \paragraph{Theory Questions}\label{theory-questions}

    1.Symbolize the following proposition and discuss the truth. 1. Everyone
has black hair. 2. Some people boarded the moon. 3. No one has boarded
Jupiter 4. Students studying in the US are not necessarily Asians.

    \emph{your answer here...}

\begin{center}\rule{0.5\linewidth}{\linethickness}\end{center}

    2.Judge the following formula, which is tautology? What is the
contradiction? 1. ∀xF(x)⇒(∃x∃yG(x,y))⇒∀xF(x)) 2. ¬(
∀xF(x)⇒∃yG(y))∧∃yG(y) 3. ∀x(F(x)⇒G(y))

    \emph{your answer here...}

\begin{center}\rule{0.5\linewidth}{\linethickness}\end{center}

    3.Which of the following are correct? 1. False \textbar{}=True. 2. (A ∧
B) \textbar{}= (A ⇔ B). 3. (A ∧ B) ⇒ C \textbar{}= (A ⇒ C) ∨ (B ⇒ C). 4.
(A ∨ B) ∧ (¬C ∨¬D ∨ E) \textbar{}= (A ∨ B). 5. (A ∨ B) ∧ (¬C ∨¬D ∨ E)
\textbar{}= (A ∨ B) ∧ (¬D ∨ E).

    \emph{your answer here...}

\begin{center}\rule{0.5\linewidth}{\linethickness}\end{center}

    4.Conjunctive normal
form.link:https://baike.baidu.com/item/\%E5\%90\%88\%E5\%8F\%96\%E8\%8C\%83\%E5\%BC\%8F/2459360
1. Obtaining conjunctive paradigm: P∧(Q⇒R)⇒S \#\#\#\# Basic steps to
find a conjunctive normal form. 1. Cut redundant connectives,Reserved
\{∨,∧,¬\} 2. Move or remove the negation \textasciitilde{} 3.
distribution rates

    \emph{your answer here...}

\begin{center}\rule{0.5\linewidth}{\linethickness}\end{center}

    5.Arithmetic assertions can be written in first-order logic with the
predicate symbol \textless{},the function symbols + and ×, and the
constant symbols 0 and 1. Additional predicates can also be defined with
biconditionals.(Chapter 8.20) 1. Represent the property ``x is an even
number.'' 2. Represent the property ``x is prime.'' 3. Goldbach's
conjecture is the conjecture (unproven as yet) that every even number is
equal to the sum of two primes. Represent this conjecture as a logical
sentence.

    \emph{your answer here...}

\begin{center}\rule{0.5\linewidth}{\linethickness}\end{center}

    \subsubsection{Programming Excercises}\label{programming-excercises}

    \begin{enumerate}
\def\labelenumi{\arabic{enumi}.}
\tightlist
\item
  Take the multiagent folder from assignment 2 and copy into a new
  directory for this practical. We will implement a version of minimax
  for Ghost Agents to make very smart ghosts.
\end{enumerate}

First look at the file ghostAgents.py. Try to play classic pacman
against the directional ghost and the random ghost. Use the following
option:

-g TYPE, -\/-ghosts=TYPE the ghost agent TYPE in the ghostAgents module
to use {[}Default: RandomGhost{]}

Now implement a new ghost agent called MinimaxGhost. You will need to
create a new class and methods in the file ghostAgent (you can ask the
tutor if you need help to create the method stub).

This will involve implementing the following methods:

To test whether or not you have implemented correctly you will need to
compare the behaviour of your new ghost agent with the random ghost
agent. This has to be done over multiple tests (e.g. 10 runs).

\begin{itemize}
\tightlist
\item
  You can turn off the graphical display to use options -t or -q
\item
  You can perform multiple runs (e.g. 10) by setting the -numGames=10
\item
  You can fix the random seed as well.
\end{itemize}

Try to use different layouts to test. A simple layout might be good for
testing initially as you are developing your algorithm. New layout files
can be created in the layout directory (must be saved with end of file
name ".lay"). For example: smallClassic.lay
\%\%\%\%\%\%\%\%\%\%\%\%\%\%\%\%\%\%\%\% \%......\%G G\%......\%
\%.\%\%...\%\% \%\%...\%\%.\% \%.\%o.\%........\%.o\%.\%
\%.\%\%.\%.\%\%\%\%\%\%.\%.\%\%.\% \%........P.........\%
\%\%\%\%\%\%\%\%\%\%\%\%\%\%\%\%\%\%\%\% newLayout.lay: \%\%\%\%\%\%\%
\% \%G\% \% \% \% \% \% \% \% \% \% \%\%\%P\%\%\% \% \% \% \%\%\% \% \%
\% \% \% \% \% \% \% \% \%\%\% \% \% . \% \%\%\%G\%\%\% \%\%\%\%\%\%\%

Please try to make some different layouts such as in this example,(the
ghosts are marked by G, walls by \% and pacman starting position by
P,"." for pellet)

\begin{center}\rule{0.5\linewidth}{\linethickness}\end{center}

2. In this question we will test the new ghostAgent with different
pacman agents. In your assignment you were asked to complete a Minimax
pacman a version of Expectimax for pacman was provided. If you have not
completed the assignment just use the provided expectimax version.

We will perform some experiments to compare the performance of pacman
against different types of ghost agent: - a random (not smart) ghost vs
a pacman that assumes optimal play from ghosts (ie minimax pacman). - a
smart (minmax) ghost vs a pacman that assumes optimal play from ghosts -
random ghosts vs pacman that assumes ghosts may not always do an optimal
move - smart (minimax) ghosts vs pacman that assumes that ghosts may do
suboptimal moves.

This will result in a table similar to the following:

3. Describe the performance (in terms of the distribution) of Pacman in
each case.

In which cases is the Pacman agent implementing the correct assumption
of the ghosts behaviour?

4. Describe why the ghosts seem as if they are cooperating when using
minimax even though they are not sharing information with each other.

    \begin{Verbatim}[commandchars=\\\{\}]
{\color{incolor}In [{\color{incolor} }]:} \PY{l+s+sd}{\PYZdq{}\PYZdq{}\PYZdq{}}
        \PY{l+s+sd}{Answer of 1.}
        \PY{l+s+sd}{\PYZdq{}\PYZdq{}\PYZdq{}}
        \PY{k}{def} \PY{n+nf}{minimax}\PY{p}{(}\PY{n+nb+bp}{self}\PY{p}{,} \PY{n}{state}\PY{p}{,} \PY{n}{sindex}\PY{p}{,} \PY{n}{depth}\PY{p}{,} \PY{n}{agent\PYZus{}num}\PY{p}{)}\PY{p}{:}
            \PY{k}{if} \PY{n}{depth} \PY{o}{\PYZgt{}}\PY{o}{=} \PY{n+nb+bp}{self}\PY{o}{.}\PY{n}{depth} \PY{o}{*} \PY{n}{agent\PYZus{}num} \PY{o+ow}{or} \PY{n}{state}\PY{o}{.}\PY{n}{isWin}\PY{p}{(}\PY{p}{)} \PY{o+ow}{or} \PY{n}{state}\PY{o}{.}\PY{n}{isLose}\PY{p}{(}\PY{p}{)}\PY{p}{:}
                \PY{k}{return} \PY{n}{betterEvaluationFunctionGhost}\PY{p}{(}\PY{n}{s}\PY{p}{,} \PY{n}{sindex}\PY{p}{)}
            \PY{k}{if} \PY{n}{depth} \PY{o}{\PYZpc{}} \PY{n}{agent\PYZus{}num} \PY{o}{==} \PY{l+m+mi}{2}\PY{p}{:}
                \PY{k}{return} \PY{n+nb+bp}{self}\PY{o}{.}\PY{n}{mini}\PY{p}{(}\PY{n}{state}\PY{p}{,} \PY{n}{depth}\PY{p}{,} \PY{n}{sindex}\PY{p}{)}
            \PY{k}{else}\PY{p}{:}
                \PY{k}{return} \PY{n+nb+bp}{self}\PY{o}{.}\PY{n}{maxi}\PY{p}{(}\PY{n}{state}\PY{p}{,} \PY{n}{depth}\PY{p}{,} \PY{n}{sindex}\PY{p}{)}
        
        \PY{k}{def} \PY{n+nf}{mini}\PY{p}{(}\PY{n+nb+bp}{self}\PY{p}{,} \PY{n}{state}\PY{p}{,} \PY{n}{depth}\PY{p}{,} \PY{n}{idx}\PY{p}{)}\PY{p}{:}
            \PY{n}{v} \PY{o}{=} \PY{n+nb}{float}\PY{p}{(}\PY{l+s+s1}{\PYZsq{}}\PY{l+s+s1}{inf}\PY{l+s+s1}{\PYZsq{}}\PY{p}{)}
            \PY{k}{for} \PY{n}{action} \PY{o+ow}{in} \PY{n}{state}\PY{o}{.}\PY{n}{getLegalActions}\PY{p}{(}\PY{l+m+mi}{0}\PY{p}{)}\PY{p}{:}
                \PY{k}{if} \PY{n}{action} \PY{o}{==} \PY{n}{Directions}\PY{o}{.}\PY{n}{STOP}\PY{p}{:}
                    \PY{k}{continue}
                \PY{n}{successor} \PY{o}{=} \PY{n}{state}\PY{o}{.}\PY{n}{generateSuccessor}\PY{p}{(}\PY{l+m+mi}{0}\PY{p}{,} \PY{n}{action}\PY{p}{)}
                \PY{n}{t\PYZus{}v} \PY{o}{=} \PY{n+nb+bp}{self}\PY{o}{.}\PY{n}{minimax}\PY{p}{(}\PY{n}{successor}\PY{p}{,} \PY{n}{idx} \PY{o}{+} \PY{l+m+mi}{1}\PY{p}{,} \PY{n}{depth}\PY{p}{)}
                \PY{n}{v} \PY{o}{=} \PY{n+nb}{min}\PY{p}{(}\PY{n}{t\PYZus{}v}\PY{p}{,} \PY{n}{v}\PY{p}{)}
            \PY{k}{return} \PY{n}{v}
        
        \PY{k}{def} \PY{n+nf}{maxi}\PY{p}{(}\PY{n+nb+bp}{self}\PY{p}{,} \PY{n}{state}\PY{p}{,} \PY{n}{depth}\PY{p}{,} \PY{n}{idx}\PY{p}{)}\PY{p}{:}
            \PY{n}{v} \PY{o}{=} \PY{n+nb}{float}\PY{p}{(}\PY{l+s+s2}{\PYZdq{}}\PY{l+s+s2}{\PYZhy{}inf}\PY{l+s+s2}{\PYZdq{}}\PY{p}{)}
            \PY{n}{actv} \PY{o}{=} \PY{k+kc}{None}
            \PY{k}{for} \PY{n}{action} \PY{o+ow}{in} \PY{n}{state}\PY{o}{.}\PY{n}{getLegalActions}\PY{p}{(}\PY{n}{idx}\PY{p}{)}\PY{p}{:}
                \PY{k}{if} \PY{n}{action} \PY{o}{==} \PY{n}{Directions}\PY{o}{.}\PY{n}{STOP}\PY{p}{:}
                    \PY{k}{continue}
                \PY{n}{successor} \PY{o}{=} \PY{n}{state}\PY{o}{.}\PY{n}{generateSuccessor}\PY{p}{(}\PY{n}{idx}\PY{p}{,} \PY{n}{action}\PY{p}{)}
                \PY{n}{t\PYZus{}v} \PY{o}{=} \PY{n+nb+bp}{self}\PY{o}{.}\PY{n}{minimax}\PY{p}{(}\PY{n}{successor}\PY{p}{,} \PY{n}{idx} \PY{o}{+} \PY{l+m+mi}{1}\PY{p}{,} \PY{n}{depth}\PY{p}{)}
                \PY{k}{if} \PY{n}{t\PYZus{}v} \PY{o}{\PYZgt{}} \PY{n}{v}\PY{p}{:}
                    \PY{n}{v} \PY{o}{=} \PY{n}{t\PYZus{}v}
                    \PY{n}{actv} \PY{o}{=} \PY{n}{action}
            \PY{k}{return} \PY{n}{actv} \PY{k}{if} \PY{n}{depth} \PY{o}{==} \PY{l+m+mi}{0} \PY{k}{else} \PY{n}{v}
        
        \PY{k}{def} \PY{n+nf}{getDistribution}\PY{p}{(}\PY{n+nb+bp}{self}\PY{p}{,} \PY{n}{state}\PY{p}{)}\PY{p}{:}
            \PY{n}{scores} \PY{o}{=} \PY{p}{[}\PY{p}{]}
            \PY{n}{agent\PYZus{}num} \PY{o}{=} \PY{l+m+mi}{3}
            \PY{n+nb+bp}{self}\PY{o}{.}\PY{n}{minimax}\PY{p}{(}\PY{n}{state}\PY{p}{,} \PY{n+nb+bp}{self}\PY{o}{.}\PY{n}{index}\PY{p}{,} \PY{l+m+mi}{0}\PY{p}{,} \PY{n}{agent\PYZus{}num}\PY{p}{)}
            \PY{n}{directions} \PY{o}{=} \PY{n}{state}\PY{o}{.}\PY{n}{getLegalActions}\PY{p}{(}\PY{n+nb+bp}{self}\PY{o}{.}\PY{n}{index}\PY{p}{)}\PY{p}{[}\PY{n}{scores}\PY{o}{.}\PY{n}{index}\PY{p}{(}\PY{n+nb}{max}\PY{p}{(}\PY{n}{scores}\PY{p}{)}\PY{p}{)}\PY{p}{]}
            \PY{n}{dist} \PY{o}{=} \PY{n}{util}\PY{o}{.}\PY{n}{Counter}\PY{p}{(}\PY{p}{)}
            \PY{k}{for} \PY{n}{a} \PY{o+ow}{in} \PY{n}{state}\PY{o}{.}\PY{n}{getLegalActions}\PY{p}{(}\PY{n+nb+bp}{self}\PY{o}{.}\PY{n}{index}\PY{p}{)}\PY{p}{:}
                \PY{n}{dist}\PY{p}{[}\PY{n}{a}\PY{p}{]} \PY{o}{=} \PY{l+m+mi}{0}
            \PY{n}{dist}\PY{p}{[}\PY{n}{directions}\PY{p}{]} \PY{o}{=} \PY{l+m+mf}{1.0}
            \PY{n}{dist}\PY{o}{.}\PY{n}{normalize}\PY{p}{(}\PY{p}{)}
            \PY{k}{return} \PY{n}{dist}
        
        
        \PY{k}{def} \PY{n+nf}{betterEvaluationFunctionGhost}\PY{p}{(}\PY{n}{state}\PY{p}{,} \PY{n}{index}\PY{p}{)}\PY{p}{:}
            \PY{l+s+sd}{\PYZdq{}\PYZdq{}\PYZdq{}}
        \PY{l+s+sd}{        Ghost evaluation function}
        \PY{l+s+sd}{    \PYZdq{}\PYZdq{}\PYZdq{}}
            \PY{n}{dis} \PY{o}{=} \PY{n}{manhattanDistance}\PY{p}{(}\PY{n}{state}\PY{o}{.}\PY{n}{getPacmanPosition}\PY{p}{(}\PY{p}{)}\PY{p}{,} \PY{n}{state}\PY{o}{.}\PY{n}{getGhostPosition}\PY{p}{(}\PY{n}{index}\PY{p}{)}\PY{p}{)}
            \PY{k}{return} \PY{n}{dis}
\end{Verbatim}



    % Add a bibliography block to the postdoc
    
    
    
    \end{document}
